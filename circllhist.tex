\documentclass{article}

\usepackage{arxiv}

\usepackage[utf8]{inputenc} % allow utf-8 input
\usepackage[T1]{fontenc}    % use 8-bit T1 fonts
\usepackage{hyperref}       % hyperlinks
\hypersetup{hidelinks = true}
\usepackage{url}            % simple URL typesetting
\usepackage{booktabs}       % professional-quality tables
\usepackage{amsfonts}       % blackboard math symbols
\usepackage{amsmath}
\usepackage{amssymb}
\usepackage{nicefrac}       % compact symbols for 1/2, etc.
\usepackage{microtype}      % microtypography
\usepackage{minted}

\usepackage{graphicx}
\usepackage{subcaption}

% ENVIRONMENTS
\usepackage{amsthm}
\theoremstyle{plain}
\newtheorem{definition}{Definition}[section]
\newtheorem{lemma}[definition]{Lemma}
\newtheorem{proposition}[definition]{Proposition}
\newtheorem{corollary}[definition]{Corollary}
\theoremstyle{remark}
\newtheorem{remark}[definition]{Remark}
\newtheorem{example}[definition]{Example}

% TEXT-MODE MACROS
\newcommand{\IN}{\mathbb{N}}
\newcommand{\IR}{\mathbb{R}}
\newcommand{\IZ}{\mathbb{Z}}
\newcommand{\union}{\cup}
\newcommand{\Union}{\bigcup}
\newcommand{\code}{\texttt} % for use as \code{test()}
\newcommand{\defn}{\emph} % for use as \defn{...}

% MATH-MODE MACROS
\newcommand{\qtext}[1]{\quad\text{#1}\quad} % for use as \def{...} inside text
\newcommand{\lra}{\longrightarrow}
\newcommand{\floor}[1]{\lfloor#1\rfloor}
\newcommand{\abs}[1]{|#1|}
\newcommand{\eps}{\epsilon}


\title{circllhist}

\subtitle{The Circonus Log-Linear Histogram}

\author{
  Heinrich Hartmann \\
  \texttt{heinrich.hartmann@circonus.com} \\
  Circonus \\
  \And
  Theo Schlossnagle \\
  \texttt{theo.schlossnagle@circonus.com} \\
  Circonus
}


\begin{document}

\maketitle

\begin{abstract}
  The circllhist histogram is a simple, fast and memory efficient data structure for capturing
  and processing large number of samples, that is particularly suited for applications in
  IT infrastructure monitoring.

  The circllhist allows arbitrary merging of pre-aggregated data without additional loss of accuracy,
  and the approximation of percentiles with low expected error and a-priori bounded maximal error.

  Open-source implementations are available for C/lua/python/Go/Java/JavaScript.
\end{abstract}

\tableofcontents

\section{Introduction}
The circllhist is a histogram data structure that allows the representation of an virtually
unlimited amount of data with bounded memory consumption, and precise a-priori bounds for the
accuracy of derived statistics.

In this note we will describe the data structure ...

\section{Related Work}

\subsection{HDR Histograms}

\subsection{t-digest}

\subsection{DD-sketches}

\section{Abstract Histograms}

\subsection{Binnings}

\begin{definition}
  An binning $B$ is a set of disjoint intervals (``bins'') $B[i], i\in I$, indexed by a
  finite or countably infinite set $I$.
  The union of all bins in a binning is called the domain, $Dom(B)$.
  \begin{align*}
    \Union_{i \in I} B[i] = Dom(B) \subset \IR,\quad B[i] \cap B[j] = \emptyset \qtext{for} i \neq j \in I.
  \end{align*}
  The map, that associates to $x \in Dom(B)$ the index $i$ of the unique bin with $x \in B[i]$,
  is called \defn{binning map}
  and denoted as follows:
  \begin{align*}
    bin: Dom(B) \lra I, x \mapsto bin(x) \qtext{with} x \in B[bin(x)]
  \end{align*}
\end{definition}

\begin{example}
  The linear binning of $\IR$ is given by $I = \IZ$, $B_i = [i, i+1)$, with $bin(x)=\floor{x}$.
\end{example}

\begin{example}
  The log-linear binning with basis $b$ of $\IR_{>0}$ is given by $I=\IZ$, $B_i = [b^i, b^{i+1})$, with $bin(x)=\floor{\log_b(x)}$.
\end{example}

\begin{remark}
  The binning map $Dom(B) \lra I$ uniquely determines the binning $B$, via
  \begin{align*}
    B[i] = bin^{-1}\{ i \} = \{ x \in Dom(B) \,|\, bin(x) = i \}.
  \end{align*}
  Given a map $f:\IR \supset D \lra I$, there is a unique binning with binning map $f$, if the fibers $f^{-1}\{i\}, i \in I$ are intervals.
\end{remark}

\subsection{Log-Linear Binnings}

\begin{figure}
  \includegraphics[width=\textwidth]{assets/LLBins.png}
  \caption{Construction of the Log-Linear Binning}

\end{figure}


\newcommand{\float}{\mathrm{float}}
\newcommand{\bin}{\mathrm{bin}}

Let $b,p \in \IN$ be integers with $b\geq 2, p \geq 1$.
The \defn{base-$b$ precision-$p$ log-linear binning} has a bin boundary at each base-$b$ floating point number with $p$ significant digits.
We will be mainly interested in the decimal precision-2 binning, which has it's bin boundaries at the base-10 floating point numbers with two significant digits.

Recall that floating point numbers are represented by sign, mantissa and exponent.
For example the number
\begin{align*}
  \underbrace{+}_{\text{sign}} \underbrace{6.6}_{\text{mantissa}} \cdot {\underbrace{10}_{\text{base}}\hspace{.5em}}^{-36} \hspace{-2em}\underbrace{\hspace{.5em}}_{\text{exponent}}.
\end{align*}
has sign $s=+1$, mantissa $d=66$ and exponent $e=-36$.

\begin{definition}
  For each tuple $(s,d,e)$ of integers with $s=\pm1$ we associate a real number:
  \begin{align*}
    \float_{b,p}(s,d,e) := s \cdot \frac{d}{b^{p-1}} \cdot b^e = s \cdot d \cdot b^{e-p+1}
  \end{align*}
  By convention we set $\float_{b,p}(0,0,0)=0$.
\end{definition}

\begin{example}
  We have
  \begin{align*}
    \float_{10,2}(+1,66,-36) = 66/10 \cdot 10^{-36}  = 6.6 \cdot 10^{-36}.
  \end{align*}
  The number $0$ is represented as $\float_{b,p}(0,0,0) = 0$.
  The number $1$ is represented as $\float_{b,p}(+1,b^{p-1},0) = 1$.
\end{example}

\begin{definition} \label{hdrdef}
  The base-$b$ precision-$p$ log-linear binning has index set
  \begin{align*}
    I_{b,p} = \{\, (s,d,e) \,|\, s,d,e \in \IZ \,\text{with}\, s=d=e=0 \,\text{or}\, s=\pm1, b^{p-1} \leq d < b^p \,\}
  \end{align*}
  and bins $B[s,d,e]$ defined by
  \begin{align*}
    B[+1,d,e] &= \mathopen[\,\float_{b,p}(+1,d,e), \float_{b,p}(+1,d+1,e)\,\mathopen), \\
    B[0,0,0]  &= \{ 0 \}, \\
    B[-1,d,e] &= \mathopen(\,\float_{b,p}(-1,d+1,e), \float_{b,p}(-1,d+1,e)\,\mathopen].
  \end{align*}
\end{definition}

\begin{remark} For the largest allowable $d=b^{p}-1$ we have $(s,d+1,e) \neq I_{b,p}$, but
  \begin{align*}
    \float_{b,p}(s,d + 1,e) = \float_{b,p}(s, b^p ,e) = \float_{b,p}(s, b^{p-1}, e+1)
  \end{align*}
  with $(s, b^{p-1}, e) \in I_{b,p}$.
  So the bin boundaries can always be chosen from the index set $I_{b,p}$.
\end{remark}

It's not clear from the above definition that the bins are disjoint and cover the real axes.
The rest of this section is devoted to demonstrating this and determining the binning map.

\begin{definition}
  For $x\in\IR, x\neq 0$, we define the following functions:
  \begin{align*}
    s(x) := sign(x), \quad
    e(x) := \floor{\log_b\abs{x}}, \quad
    d(x) := \floor{\abs{x} \cdot b^{-e(x)+p-1}} \quad
    \delta(x) := \abs{x} \cdot b^{-e(x)+p-1} - d(x).
  \end{align*}
  Moreover we set $s(0) = e(0) = d(0) = \delta(0) = 0$.
\end{definition}

\begin{lemma} \label{floatlem}
  (A) For $x \neq 0$ we have
  \begin{align*}
    b^{p-1} \leq d(x) < b^p \qtext{and} 0 \leq \delta(x) < 1.
  \end{align*}

  (B) For $x \in \IR$ we have
  \begin{align*}
    x = s(x) \cdot (d(x) + \delta(x)) \cdot b^{e(x) - p + 1}
  \end{align*}

  (C) If $x \neq 0$ is represented as
  \begin{align*}
    x = s \cdot (d + \delta) \cdot b^{e - p + 1}
  \end{align*}
  with $s \in \{\pm 1\}$, $\delta \in [0,1)$ and $d \in \IZ$ with $b^{p-1} \leq d < b^p$, then
  \begin{align*}
    s = s(x), \quad d = d(x), \quad e = e(x) \qtext{and} \delta = \delta(x).
  \end{align*}
\end{lemma}

\begin{proof}
  We start with claim (B). For $x=0$ we have $s(0) = 0$ hence the equation holds.
  For $x \neq 0$, we have
  \begin{align*}
    x = sign(x) |x| = sign(x) \abs{x} \cdot b^{-e(x)+p-1} \cdot b^{e(x)-p+1} = s(x) \cdot (d(x) + \delta(x)) \cdot b^{e(x)-p+1}
  \end{align*}
  as claimed.

  Ad A) Note that $\delta(x)$ is of the form $y - \floor{y}$ and hence in $[0,1)$.

  For $x \neq 0$, we write $|x| = b^{log_b|x|} = b^{e(x) + \eps}$, with $\eps \in [0,1)$.
  Hence $\abs{x} / b^{e(x)} = b^\eps \in [1,b)$ and
  \begin{align*}
     d(x) + \delta(x) = \abs{x} \cdot b^{-e(x)+p-1} = \abs{x} / b^{e(x)} \cdot b^{p-1} \in [b^{p-1},b^p).
  \end{align*}
  It follows that also $d(x) = \floor{d(x) + \delta(x)} \in [ b^{p-1}, b^p )$
  since both $b^{p-1}$ and $b^p$ are integers.

  Ad C) We have
  \begin{align*}
    s(x) &= sign(s \cdot (d + \delta) \cdot b^{e-p+1}) = s, \\
    e(x) &= \floor{ \log_b \abs{s \cdot (d + \delta) \cdot b^{e-p+1}} } = \floor{\log_b((d+\delta)/b^{p-1})} + e = e, \\
    d(x) &= \floor{\abs{ s \cdot (d + \delta) \cdot b^{e-p+1} } \cdot b^{p-1-e(x)}} = \floor{d + \delta} = d, \\
    \delta(x) &= (d + \delta) - \floor{d + \delta} = \delta.
  \end{align*}
  In the second equation we used, that $b^{p-1} \leq d < b^p$ and this inequality between integers
  not changed by adding a small real number $\delta \in [0,1)$ to $d$.
  Hence $1 \leq (d+\delta) /  b^{p-1} < b$ and $0 \leq log_b((d+\delta) /  b^{p-1}) < 1$.
\end{proof}

\begin{corollary}
  A number $x \in \IR$ is of the form
  \begin{align*}
    x = \float_{b,p}(s,d,e)
  \end{align*}
  for some integers $s,d,e$ with $b^{p-1} \leq d < b^p$, $s \in \{\pm 1\}$, if and only if $\delta(x) = 0$.
  In this case
  \begin{align*}
    s = s(x), \quad d = d(x), \quad e = e(x).
  \end{align*}
\end{corollary}
\begin{proof}
  By the Lemma \ref{floatlem} (B), we have
  \begin{align*}
    x = s(x) (d(x) + \delta(x)) b^{e(x)-p+1} = \float_{b,p}(s(x), d(x), e(x)) + \delta(x) \cdot b^{e(x) - p + 1},
  \end{align*}
  so $\delta(x) = 0$ implies $x = \float_{b,p}(s(x),d(x),e(x))$.

  Conversely, if $x = \float_{b,p}(s,d,e)$ then $s = s(x), d = d(x), e = e(x), \delta(x) = 0$ by \ref{floatlem} (C).
\end{proof}

\begin{proposition} \label{hdrprop}
  For $x \in \IR, (s,d,e) \in I_{b,p}$ we have $x \in B[s,d,e]$ if and only if $\bin_{b,p}(x) = (s,d,e)$.
\end{proposition}
\begin{proof}
  Assume that $x \in B[+1,d,e]$. By definition of $B[1,d,e]$ we have
  \begin{align*}
    \float_{b,p}(+1,d,e) = & d \cdot b^{e-p+1} \leq x < \float_{b,p}(+1,d+1, e) = (d+1) \cdot b^{e-p+1}.
  \end{align*}
  Dividing by $b^{e-p+1}$ we find that $\delta := x/b^{e-p+1} - d$ lies in $[0,1)$.
  Solving for $x$ we find $x = (d + \delta) \cdot b^{e-p+1}$, and it follows by
  Lemma \ref{floatlem} (C), that $d=d(x), e=e(x), \delta = \delta(x)$, so $\bin_{b,p}(x) = (1,d,e)$.

  Conversely if $\bin_{b,p}(x) = (1,d,e)$, then $x = (d + \delta(x)) \cdot b^{e - p + 1}$
  \begin{align*}
      \float_{b,p}(1,d,e) = (d + 0) \cdot b^{e - p + 1} \leq (d + \delta(x)) \cdot b^{e - p + 1} <  (d + 1) \cdot b^{e - p + 1} = \float_{b,p}(1,d+1,e)
  \end{align*}
  and hence $x \in B[1,d,e]$.

  For $x = 0$ we have $bin(x) = (0,0,0)$ by definition.
  Conversely if $bin(x) = (0,0,0)$ then $s(x) = 0$ so $x = 0$.

  The case $x < 0$ can be derived from the first equation using the identities $bin(-x)=(-s(x),d(x),e(x))$ and $B[-1,d,e] = -B[1,d,e]$.
\end{proof}

\begin{corollary}
  (A) The log-linear bins $B[s,d,e]$ are precisely the fibers of the map
  \begin{align*}
    bin_{b,p}: \IR \lra I_{b,p}, \quad x \mapsto bin_{b,p}(x) = (s(x), d(x), e(x) ),
  \end{align*}
  i.e. $\bin_{b,p}^{-1}\{(s,d,e)\} = B[s,d,e].$

  (B) The bins $B[s,d,e], (s,d,e) \in I_{b,p}$ are disjoint and collectively cover the real axes $\IR$.

  (C) log-linear histograms are well defined by Definition \ref{hdrdef} and have bin mapping $\bin_{b,p}: \IR \lra I_{b,p}$.

  (D) The bin sizes for the log-linear binning is given by $size(B[s,d,e]) = b^{e-p+1}$.
\end{corollary}

\begin{proof}
  (A) is a reformulation of Proposition \ref{hdrprop}.
  (B) follows from (A) since the fibers $f^{-1}\{x\}$ of any map $f:X \rightarrow Y$ are disjoint and cover the domain $X$.
  (C) follows from (B) and (A).

  (D) follows directly from the definition since $\float_{b,p}(1,d+1,e)-\float_{b,p}(1,d,e) = b^{e-p+1}$.
\end{proof}

\subsection{Histogram Summaries}
Given a binning $B_i, i\in I$ of $D \subset \IR$, and  dataset $X=(x_1,\dots, x_N) $ with values in
$D \subset \IR$, we associate a count function:

\begin{align*} c_D(i) = \# \{ j | y_j \in B_i \} \end{align*}

The datum of a binning $B_i,i\in I$ and a count function $c_i,\in I$, is called a histogram summary of $X$.

\subsection{Imlementation}

\begin{itemize}
\item HdrHistogram
\item t-digest
\end{itemize}

\section{The Circllhist Datastructure}
\begin{minted}[frame=lines,framesep=2mm]{c}

struct histogram {
  uint16_t allocd;
  uint16_t used;
  struct hist_bv_pair *bvs;
};

struct hist_bv_pair {
  hist_bucket_t bucket;
  uint64_t count;
};

typedef struct hist_bucket {
  int8_t val;
  int8_t exp;
} hist_bucket_t;
\end{minted}

\section{Histogram Operations}

\subsection{Merging}

\subsection{Mean Values}

\subsection{Percentiles}

\section{Evaluation}

In this section compare the following quantile sketches:

\begin{itemize}
\item[exact]
  Exact quantile estimation based on numpy arrays \url{numpy.org}
\item[prom]
  Quantile estimation based on \href{prometheus.io}{Prometheus} Histograms.
  We use a hand-written Python port of the original quantile functions written in go.
\item[hdr]
  The High Dynamic Range Histogram data-structure was introduced in \cite{hdr}.
  We use the Python implementation published at \url{https://github.com/HdrHistogram/HdrHistogram_py}.
\item[t-digest]
  The t-digest data structure was introduced in \cite{tdigest}.
  We use Java implementation by the original authors available at \url{github.com/tdunning/t-digest}.
\item[dd]
  The ``Distributed Distribution Sketch'' is a histogram datastructure that was introduced in \cite{dd}.
  We use the Python implementation published at \url{github.com/DataDog/sketches-py}
\item[circllhist]
  The Circonus Log-Linear Histogram datastructure described in this document.
  We use the Python/C implementation published at \url{github.com/circonus-labs/libcircllhist}.
\end{itemize}

The Prometheus monitoring system makes use of a very simple quantile sketch that consists of a list
of ``less than'' metrics, which count how many samples were inserted that are below manually
configured threshold values.
Prometheus Histograms are in wide use in practice.
The evaluation results are highly dependent on the number and location of the chosen thresholds.
We use the recommended number of around 10 threshold values, at locations which cover the whole
data range with emphasis on the likely quantile locations.
In practice these thresholds have to be chosen without knowledge of the dataset that is being recorded.

The HDR Histogram method was configured to resemble the circllhist method, with a range of
$10^{-128} .. 10^{128}$ and a accuracy of three significant digits.



\clearpage
\subsection{Calibration}

\begin{figure}
   \includegraphics[width=\textwidth]{evaluation/images/quantile_comparison.png}
   \includegraphics[width=\textwidth]{evaluation/images/Two_Points_quantile_comparison.png}
   \caption{Percentile functions on two-point dataset}
\end{figure}

\begin{figure}
  \centering
  \begin{tabular}{lrrrrrrr}
\toprule
{} &   exact &    prom &  tdigest &     hdr &      dd &  circllhist/type-1 &  circllhist/type-7 \\
\midrule
q0  &  10.000 &   9.900 &   10.000 &   9.999 &  10.000 &             10.500 &             10.500 \\
q.1 &  10.000 &   9.930 &   10.000 &  10.006 &  10.075 &             10.500 &             10.500 \\
q.2 &  10.000 &   9.960 &   10.000 &  10.006 &  10.075 &             10.500 &             10.500 \\
q.3 &  10.000 &   9.990 &   19.000 &  10.006 &  10.075 &             10.500 &             10.500 \\
q.4 &  10.000 &  10.020 &   37.000 &  10.006 &  10.075 &             10.500 &             10.500 \\
q.5 &  10.000 &  10.050 &   55.000 &  10.006 &  10.075 &             10.500 &             10.500 \\
q.6 &  10.000 &  99.930 &   73.000 &  10.006 &  10.075 &            105.000 &             10.500 \\
q.7 &  10.000 &  99.960 &   91.000 &  10.006 &  10.075 &            105.000 &             10.500 \\
q.8 &  10.000 &  99.990 &  100.000 & 100.019 &  10.075 &            105.000 &             10.500 \\
q.9 &  10.000 & 100.020 &  100.000 & 100.019 &  10.075 &            105.000 &             10.500 \\
q1  & 100.000 & 100.050 &  100.000 & 100.019 & 100.000 &            105.000 &            105.000 \\
\bottomrule
\end{tabular}

  \caption{Percentile values of the two-point dataset $[10,100]$}
\end{figure}

\clearpage
\subsection{Datasets}

\begin{figure}
   \includegraphics[width=\textwidth]{evaluation/images/Uniform_Distribution_distribution_percentiles.png}
   \includegraphics[width=\textwidth]{evaluation/images/API_Latencies_distribution_percentiles.png}
   \includegraphics[width=\textwidth]{evaluation/images/Simulated_Latencies_distribution_percentiles.png}
   \caption{Datasets}
\end{figure}

\clearpage
\subsection{Size}

\begin{figure*}[t!]
  \includegraphics[width=\textwidth]{evaluation/images/all_size.png}
  \caption{Size Comparison}
\end{figure*}

\begin{figure}
  \centering
  \begin{tabular}{lrrrrrr}
\toprule
{} &     exact &  prom &  hdr &  tdigest &    dd &  circllhist \\
\midrule
Uniform Distribution &    800000 &    96 &  621 &     2224 &   716 &         453 \\
API Latencies        &  37278488 &    96 & 2994 &     2368 &  1902 &        1866 \\
Simulated Latencies  &   8028624 &    96 & 3699 &     2096 &  3428 &        3396 \\
\bottomrule
\end{tabular}

  \caption{Aggregation sizes in kb}
\end{figure}

\clearpage
\subsection{Performance}

\begin{figure}
  \includegraphics[width=\textwidth]{evaluation/images/Uniform_Distribution_perf.png}
  \includegraphics[width=\textwidth]{evaluation/images/API_Latencies_perf.png}
  \includegraphics[width=\textwidth]{evaluation/images/Simulated_Latencies_perf.png}
  \caption{Performance Comparison}
\end{figure}

Tables in usec

Uniform\\
\begin{tabular}{lrrrrrr}
\toprule
{} &  exact &  prom &    hdr &  tdigest &   dd &  circllhist \\
\midrule
Insertion            &    0.9 &  10.5 &    7.7 &      4.6 &  3.8 &         2.2 \\
Merge                &   33.1 &   1.1 &  129.0 &     26.9 & 45.8 &         6.3 \\
Quantile Computation & 1180.3 &  11.3 & 9684.9 &      1.4 & 17.1 &         3.8 \\
\bottomrule
\end{tabular}


API Latencies\\
\begin{tabular}{lrrrrrr}
\toprule
{} &   exact &  prom &    hdr &  tdigest &    dd &  circllhist \\
\midrule
Insertion            &     0.0 &   6.9 &    3.7 &      2.0 &   3.0 &         1.0 \\
Merge                &  3487.6 &   0.9 &   34.1 &     24.0 & 104.6 &        16.1 \\
Quantile Computation & 83773.2 &   8.3 & 1931.4 &      1.1 &  42.6 &         6.1 \\
\bottomrule
\end{tabular}


Simulated API Latency Data\\
\begin{tabular}{lrrrrrr}
\toprule
{} &   exact &  prom &    hdr &  tdigest &    dd &  circllhist \\
\midrule
Insertion            &     0.1 &   6.5 &    3.2 &      1.9 &   2.5 &         1.0 \\
Merge                &   415.4 &   1.0 &   37.1 &     27.1 & 185.7 &        31.7 \\
Quantile Computation & 14222.5 &  10.0 & 1583.0 &      1.1 &  51.4 &         8.9 \\
\bottomrule
\end{tabular}


\clearpage
\subsection{Accuracy}

\begin{figure}
  \includegraphics[width=\textwidth]{evaluation/images/Uniform_Distribution_accuracy.png}
  \includegraphics[width=\textwidth]{evaluation/images/API_Latencies_accuracy.png}
  \includegraphics[width=\textwidth]{evaluation/images/Simulated_Latencies_accuracy.png}
  \caption{Accuracy Comparison}
\end{figure}

\begin{figure}
  Uniform\\
  \begin{tabular}{lrrrrr}
\toprule
{} &  prom &   hdr &  tdigest &    dd &  circllhist \\
\midrule
q0-err\%  & 0.014 & 0.156 &    0.000 & 0.000 &       0.005 \\
q.1-err\% & 0.339 & 0.248 &    0.129 & 0.342 &       0.160 \\
q.2-err\% & 0.018 & 0.178 &    0.013 & 0.736 &       0.069 \\
q.3-err\% & 0.006 & 0.273 &    0.036 & 0.794 &       0.024 \\
q.4-err\% & 0.074 & 0.297 &    0.003 & 0.394 &       0.048 \\
q.5-err\% & 0.000 & 0.243 &    0.040 & 0.216 &       0.007 \\
q.6-err\% & 0.094 & 0.468 &    0.006 & 0.931 &       0.015 \\
q.7-err\% & 0.105 & 0.006 &    0.013 & 0.050 &       0.000 \\
q.8-err\% & 0.004 & 0.010 &    0.001 & 0.307 &       0.009 \\
q.9-err\% & 0.017 & 0.053 &    0.006 & 0.081 &       0.000 \\
q1-err\%  & 0.000 & 0.073 &    0.000 & 0.000 &       0.001 \\
\bottomrule
\end{tabular}

  API Latencies\\
  \begin{tabular}{lrrrrr}
\toprule
{} &    prom &   hdr &  tdigest &    dd &  circllhist \\
\midrule
q0-err\%      & 100.000 & 0.038 &    0.000 & 0.000 &       0.134 \\
q.25-err\%    &  52.671 & 0.035 &    0.216 & 0.460 &       0.000 \\
q.5-err\%     & 126.469 & 0.038 &    0.000 & 0.538 &       0.003 \\
q.75-err\%    & 145.163 & 0.061 &    0.049 & 0.171 &       0.001 \\
q.9-err\%     &   8.230 & 0.007 &    1.382 & 0.685 &       0.006 \\
q.95-err\%    &  10.636 & 0.023 &    0.907 & 0.006 &       0.019 \\
q.99-err\%    &   5.662 & 0.031 &    0.066 & 0.164 &       0.105 \\
q.995-err\%   &   7.313 & 0.064 &    0.026 & 0.399 &       0.155 \\
q.999-err\%   &   8.584 & 0.010 &    0.030 & 0.979 &       0.202 \\
q.9999-err\%  &   8.862 & 0.019 &    0.007 & 0.715 &       0.216 \\
q.99999-err\% &   8.891 & 0.050 &    0.003 & 0.683 &       0.216 \\
q1-err\%      &   8.894 & 0.047 &    0.000 & 0.000 &       0.217 \\
\bottomrule
\end{tabular}

  Simulated API Latency Data\\
  \begin{tabular}{lrrrrr}
\toprule
{} &     prom &   hdr &  tdigest &    dd &  circllhist \\
\midrule
q0-err\%      &  100.000 & 0.407 &    0.000 & 0.000 &       2.110 \\
q.25-err\%    & 1706.372 & 0.232 &    0.141 & 0.238 &       0.062 \\
q.5-err\%     & 1523.356 & 0.012 &    0.069 & 0.243 &       0.029 \\
q.75-err\%    &  857.929 & 0.054 &    0.101 & 0.957 &       0.016 \\
q.9-err\%     &  296.620 & 0.689 &    0.342 & 0.546 &       0.009 \\
q.95-err\%    &   95.263 & 0.632 &    0.813 & 0.277 &       0.004 \\
q.99-err\%    &  133.561 & 0.146 &    4.152 & 0.087 &       0.059 \\
q.995-err\%   &   13.454 & 0.029 &   10.259 & 0.649 &       0.008 \\
q.999-err\%   &    6.136 & 0.406 &   36.221 & 0.519 &       0.002 \\
q.9999-err\%  &   23.154 & 1.933 &  149.000 & 0.599 &       0.670 \\
q.99999-err\% &  355.293 & 2.224 & 2559.294 & 0.301 &       0.570 \\
q1-err\%      &   98.875 & 0.347 &    0.000 & 0.000 &       0.409 \\
\bottomrule
\end{tabular}

\end{figure}

\bibliographystyle{unsrt}
\begin{thebibliography}{1}

\bibitem{tdigest}
T. Dunning and O. Ertl.
\newblock  Computing extremeley accurate quantiles using t-digests.
\newblock \url{https://github.com/tdunning/t-digest}, 2017.

\bibitem{dd}
M. Charles, J.E. Rim, H.K. Lee.
\newblock DDSketch: A fast and fully-mergeable quantile sketch with relative-error guarantees.
\newblock Proceedings of the VLDB Endowment 12.12 (2019): 2195-2205.

\bibitem{hdr}
G. Tene,
\newblock HdrHistogram: A high dynamic range (hdr) histogram
\newblock \url{http://hdrhistogram.org/}, 2012

\bibitem{libcircllhist}
  Circonus,
  \newblock libcircllhist: An implementation of Circonus Log-Linear Histograms
  \newblock \url{https://github.com/circonus-labs/libcircllhist}, 2016

\end{thebibliography}
\end{document}
